\documentclass[10pt,a4paper]{article}

% Set the page margins.
\usepackage[a4paper,margin=0.75in]{geometry}

% Setup the language.
\usepackage[english]{babel}
\hyphenation{Some-long-word}

\usepackage{resume}

\begin{document}
\sloppy  % this to relax whitespacing in favour of straight margins

\maintitle{.: Andrew Blinn}{ }{\today}
  %\hspace{0.1cm}\faCode
  %\faHeart
  %\faClone
  %\faGamepad \hspace{0.1cm}
  %\faAreaChart \hspace{0.1cm}
  %\faBomb
  %\faFilter
  %\faCloud 
  %\faObjectGroup
  %\faPaperclip

\nobreakvspace{0.5em} 

\noindent
    %\textsmaller+1 (734) 492-2543\sbull
    \href{http://andrewblinn.com}{andrewblinn.com}
    \hspace{0.1cm} \faEnvelopeSquare \hspace{0.1cm} 
    \href{mailto:me@andrewblinn.com}{me\mbox{}@\mbox{}andrewblinn.com}
    \hspace{0.1cm} \faGithub \hspace{0.1cm}
    \href{http://github.com/disconcision}{github.com/disconcision}
    \hspace{0.1cm} \faTwitter \hspace{0.1cm} 
    \href{http://twitter.com/disconcision}{twitter.com/disconcision}

\roottitle{  Research Interests \hspace{0.1cm} \faCoffee} 

\bodytext{Programming Languages \sbull Human-Computer Interaction \sbull Computing Education}

\roottitle{Publications \hspace{0.1cm} \faSend}

  \headedsubsection
    {\href
      {https://ieeexplore.ieee.org/abstract/document/9833110}
      {An Integrative Human-Centered Architecture for Interactive Programming Assistants} \sbull VLHCC}
    {2022}
    {\textbf{Andrew Blinn, David Moon, Eric Griffis, Cyrus Omar} \\
     A conceptual architecture for programming assistants addressing integrative design challenges, \\ instantiated in OCaml-based prototype UIs for type \& example-based interactive program synthesis}

  \headedsubsection
    {\href
      {https://dl.acm.org/doi/abs/10.1145/3546196.3550164}
      {Tylr: A Tiny Tile-based Structure Editor} \sbull TyDe}
    {2022}
    {\textbf{David Moon, Andrew Blinn, Cyrus Omar} \\
    A new kind of structure editing, maintaining term structure while retaining linear editing affordances}
  
  \headedsubsection
    {\href
      {https://dl.acm.org/doi/10.1145/3453483.3454059}
      {Filling typed holes with live GUIs} \sbull PLDI}
    {2021}
    {\textbf{Cyrus Omar, David Moon, Andrew Blinn, Ian Voysey, Nick  Collins, Ravi Chugh} \\
    Livelits allow users to fill program holes by directly manipulating user-defined GUIs \\
    embedded persistently into code, providing continuous graphical feedback}

\roottitle{Education \hspace{0.1cm} \faGraduationCap}

  \headedsubsection
    {\href
      {https://rackham.umich.edu/}
      {University of Michigan} \sbull Ph.D Candidate, Computer Science}
    {Current}
    {Researching user interfaces for/as programming languages at Cyrus Omar's \href{http://fplab.mplse.org/}{FP Lab}. \\
    Building intelligent programming interfaces combining LLMs with type-directed formal methods}
    
  \headedsubsection
    {\href
      {https://rackham.umich.edu/}
      {University of Michigan} \sbull Master’s of Science, Computer Science}
    {2023}
    {Coursework in Programming Language Theory, Category Theory, \\
    Program Synthesis, Human-Computer Interaction, \& How People Learn}
    
  \headedsubsection
    {\href
      {http://www.utoronto.ca}
      {University of Toronto} \sbull H.B.Sc, Math \& Computer Science}
    {2019}
    {Built a Racket-based x86/C compiler for a $\lambda$-calculus-based language with macro system.\\
    Graduate coursework in abstract algebra, compilers, \& graphics
    %Coursework in algorithms, concurrency, differential geometry, operating systems \& topology.
    }

\roottitle{ Industry Experience \hspace{0.1cm} \faCodeFork}

  \headedsubsection
    {\href{http://todaqfinance.net/}{TODAQ Toronto} \sbull Software Development in Clojure}
    {2019 - 2020}
    {Around front: Building novel interfaces to  \href{https://andrewblinn.com/portfolio/todaq/}{sharpen the materiality of distributed digital assets}.\\ Out back: Implementing features and services supporting a new protocol for decentralized  \\digital asset management based on a Merkel-trie-derived distributed data structure.}

\roottitle{Conference Participation \hspace{0.1cm} \faPlane}

  \headedsubsection
    {Invited speaker at RacketCon}
    {2019 \sbull Salt Lake City}
    {Spoke about \href{https://github.com/disconcision/fructure}{Fructure}, a prototype structured editor focused on edit-time term-rewriting \\
    \href{https://www.youtube.com/watch?v=CnbVCNIh1NA}{Recorded Talk} \sbull
    \href{https://github.com/disconcision/fructure/blob/master/screenshots/REAL-RacketCon-Fructure-Talk.pdf}{Fructure Slides}}

  \headedsubsection
    {Presenter at VL/HCC}
    {2022 \sbull Rome}
    {Presented an Integrative Human-Centered Architecture for Interactive Programming Assistants \\
    \href{https://www.youtube.com/watch?v=G_9Yyut3ckw}{Recorded Talk} \sbull
    \href{https://docs.google.com/presentation/d/1lrclRzlx-ayd_iqOENyvCtnexBgTpkaSltPmgQQi_IE/edit?usp=sharing}{VLHCC Slides}}
    
  \headedsubsection {Seat Filler}
    {Rome, Chicago, Salt Lake City, Toronto, Eugene, St.Louis}
    {2022, 2021: \href{https://conf.researchr.org/home/vlhcc-2022}{VL/HCC}, \href{https://2021.splashcon.org/track/splash-2021-oopsla}{SPLASH/OOPSLA} \\
    2019: \href{https://school.racket-lang.org/2019/plan/}{Racket's How to Design Languages Summer School}, \href{https://clojurenorth.com/}{Clojure North}. \\ 
    2018: \href{https://www.cs.uoregon.edu/research/summerschool/summer18/}{Oregon Programming Languages Summer School}, \href{https://conf.researchr.org/home/icfp-2018}{ICFP}, \href{https://www.thestrangeloop.com/2018/sessions.html}{Strange Loop}, \href{https://con.racket-lang.org/2018/}{RacketCon}}
    
\roottitle{Research Projects \hspace{0.1cm} \faFlask}

  \headedsubsection
    {Hazel: Experimental IDE Design, Implementation, and At-Scale Deployment}
    {2020 - Current}
    {Led a ground-up \href{https://hazel.org/build/haz3l-tests/}{rewrite of the Hazel IDE}, extending David Moon's tylr restructuring engine into a \\ full-fledged editor, language server, and educational tool which was deployed to 100 undergraduates.\\ Designed \& implemented novel bidirectional type system features. \href{https://hazel.org/}{More about the Hazel project}}

  \headedsubsection
    {Techniques in Variability-aware Data Structures with Marsha Chechik}
    {2018 - 2019}
    {Built \& profiled Haskell data structures supporting variational analysis of software product lines. \\
    Designed \& built \href{https://github.com/disconcision/spyshare}{SpyShare}, a Graphviz-based tool to visually inspect data sharing. \\
    Created and modelled a system of GHC rewrite-rules using PLT Redex:    \href{https://github.com/disconcision/vardatalab/blob/master/CSC495_TECHNIQUES_IN_VARIABILITY_AWARE_DATA_STRUCTURES.pdf}{Report} \sbull \href{https://github.com/disconcision/vardatalab/blob/master/CSC495_variational_data_structures_slides.pdf}{Slides}}
    
  \headedsubsection
    {Independent Study in Structured Editing in Racket with Gary Baumgartner}
    {Summer 2017}
    {Self-initiated study of existing refactoring, live programming \& direct manipulation tooling.\\
    Began work on \href{https://github.com/disconcision/fructure}{Fructure}, a Racket-based polyglot structure editor, and \href{https://github.com/disconcision/containment-patterns}{Containment Patterns}, \\ which extend pattern matching to capture contexts as composable continuations.}

\roottitle{Teaching \hspace{0.1cm} \faPuzzlePiece}

  \headedsubsection
    {Course Development}
    {Summer 2018 \sbull University of Toronto}
    {Designed assignments and course materials for CSC324 - Principles of Programming Languages. \\
    Specified and built \href{https://github.com/disconcision/ductile}{Ductile}, a toy language demonstrating exhaustive pattern matching on ADTs. \\
    Implemented an \href{https://github.com/disconcision/racketlab/blob/master/choice-stepper.rkt}{algebraic stepper} to illustrate continuations and non-determinism in Scheme.}
  
  \headedsubsection
    {Teaching Assistance}
    {University of Michigan}
    {\begin{tabular}{p{2.3cm} l}
    2022 & EECS490 Programming Languages \\
    2021 & EECS490 Programming Languages
    \end{tabular}}
    
  \headedsubsection
    {Teaching Assistance}
    {University of Toronto}
    {\begin{tabular}{p{2.3cm} l}
    2019, 2018 & CSC324 Principles of Programming Languages\\
    2018 & CSC104 Introduction to Computational Thinking \\
    2018, 2017 & CSC324 Principles of Programming Languages \\
    \end{tabular}}
  
\roottitle{Mentorship \hspace{0.1cm}  \faComments} 

  \headedsubsection
    {PL x Deep Reinforcement Learning for Code Completion}
    {2021 Fall - Current}
    {ML/PL collaboration: Mentoring two undergraduates, providing design and architecture guidance for a specialized version of the Hazel editor used for reinforcement-learning-based code completion research.}
  
  \headedsubsection
    {Interfaces for Live testing in online IDEs}
    {2021 - 2022}
    {Mentored an undergraduate who helped implement live testing features of the Hazel language and IDE, successfully deployed to a class of 100 undergraduate students for a course in Fall 2022.}
 
  %\roottitle{Languages}
  %\bodytext{English \sbull French}

\end{document}